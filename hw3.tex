\documentclass{article}
\usepackage[utf8]{inputenc}
\usepackage{amsmath}
\usepackage{hyperref}
\usepackage{amssymb}
\usepackage{xcolor}
\usepackage{tikz}

\newcommand{\R}{\mathbb{R}}
\newcommand{\Z}{\mathbb{Z}}
\newcommand{\N}{\mathbb{N}}
\newcommand{\Q}{\mathbb{Q}}
\newcommand{\C}{\mathbb{C}}
\newcommand{\T}{\mathbb{T}}
\newcommand{\Sc}{\mathcal{S}}

\title{Homework 3 --  Due 2nd March}
\author{\textbf{Richeek Das -- 66113700}}

\begin{document}

\maketitle

\textbf{Problem 1. } (a)
\begin{gather*}
    \delta * f(h) = \delta(\tilde{f} * h) \quad \text{ by defn}\\
    = (\tilde{f}*h)(0) \quad \text{ defn of Dirac Delta}\\
    = \int_{\R} \tilde{f}(t) h(0-t) dt\\
    = \int_{\R} f(-t) h(-t) dt \\
    = \int_{\R} f(u)h(u)du \\
    = f(h)
\end{gather*}
Here, we do abuse notation to write $f$ as Schwartz and Tempered distributions depending on the context. Hence $\delta * f(h) = f(h) \implies \delta * f = f$.
\\~

(b) 
\begin{gather*}
    \widehat{T * f}(h) = (T*f)(\hat{h}) \quad \forall h \in \Sc\\
    = T(\tilde{f} * \hat{h})
\end{gather*}
Now lets apply some properties of FT
\begin{gather*}
    \widehat{\tilde{f} * \hat{h}} = \hat{\tilde{f}} \cdot \hat{\hat{h}} = \hat{\tilde{f}} \cdot \tilde{h}\\
    \implies \tilde{f} * \hat{h} = (\hat{\tilde{f}} \cdot \tilde{h})^{\Breve{}}\\
    = \widetilde{\widehat{\hat{\tilde{f}} \cdot \tilde{h}}} \\
    = \widehat{\hat{f}\cdot h} \quad \boldsymbol{(*)}
\end{gather*}
Implies
\begin{gather*}
    T(\tilde{f} * \hat{h}) = T(\widehat{\hat{f}\cdot h}) = \hat{T}(\hat{f}\cdot h) = \hat{T}\cdot \hat{f}
\end{gather*}

Justification for $\boldsymbol{(*)}$
\begin{gather*}
    \hat{\tilde{f}}(\xi) = \int_{\R} f(-x) e^{- 2\pi i \xi x} dx = \int_{\R} f(u)e^{- 2\pi i (-\xi) u} du = \hat{f}(-\xi) = \tilde{\hat{f}}(\xi)
\end{gather*}
\\~


\textbf{Problem 2. } Since Schwartz functions with compactly supported fourier transform are dense in $L^p(\R)$, for any $f \in L^p(\R)$ and $\forall \varepsilon > 0, \exists g:$
\begin{gather*}
    \lVert f - g\rVert_p < \frac{\varepsilon}{2 C_p}
\end{gather*}
and $\exists R_0: \hat{g}(\xi) = 0$ for $|\xi| > R_0$.
\begin{gather*}
    S_R g = g, \, \forall R \geq R_0
\end{gather*}
Now we can consider the $L^p$ distance of $S_Rf $ and $f$:
\begin{gather*}
    \lVert S_R f - f \rVert_p \leq \lVert S_Rf - S_R g\rVert_p + \lVert S_Rg - g \rVert_p + \lVert g-f \rVert_p
\end{gather*}
Now, 
\begin{gather*}
    \lVert S_Rf - S_R g\rVert_p \leq C_p \lVert f-g \rVert_p < C_p \frac{\varepsilon}{2C_p} = \frac{\varepsilon}{2}\\
    \lVert S_Rg - g \rVert_p = 0, \forall R \geq R_0\\
    \lVert g-f \rVert_p < \frac{\varepsilon}{2C_p}
\end{gather*}
Therefore, 
\begin{gather*}
    \lVert S_R f - f \rVert_p < \frac{\varepsilon}{2} + 0 + \frac{\varepsilon}{2C_p}
\end{gather*}
Hence, $S_R f \to f$ in $L^p(\R)$ as $R\to \infty$.
\\~

\textbf{Problem 3. } \textbf{1. Positivity}
\begin{gather*}
    \lVert f \rVert_{p, \infty} = \sup_{\lambda > 0} \lambda \mu\{x: |f(x)| > \lambda\}^{\frac{1}{p}}
\end{gather*}
Since, supremum of non negative numbers is non negative $\lVert f \rVert_{p, \infty} \geq 0$. Now, if $\lVert f \rVert_{p, \infty} = 0$, then $\forall \lambda > 0$:
\begin{gather*}
    \lambda \mu\{x: |f(x)| > \lambda\}^{\frac{1}{p}} = 0\\
    \implies \mu\{x: |f(x)| > \lambda\} = 0 \quad \text{ since } \lambda \text{ is arbitrary }
\end{gather*}
Therefore if $\lVert f \rVert_{p, \infty} = 0$, $f=0$ a.e.
\\~

\textbf{2. Absolute Homogeneity}

For any scalar $\alpha$,
\begin{gather*}
    \lVert \alpha f \rVert_{p, \infty} = \sup_{\lambda > 0} \lambda \mu\{x: |\alpha f(x)| > \lambda\}^{\frac{1}{p}}\\
    = \sup_{\tau > 0} |\alpha| \tau \mu\{x: |\alpha f(x)| > |\alpha|\tau\}^{\frac{1}{p}} \quad \text{ substituting } \lambda = \tau |\alpha|\\
    = \sup_{\tau > 0} |\alpha| \tau \mu\{x: |f(x)| > \tau\}^{\frac{1}{p}}\\
    = |\alpha| \sup_{\tau > 0} \tau \mu\{x: |f(x)| > \tau\}^{\frac{1}{p}}\\
    = |\alpha| \lVert f \rVert_{p, \infty}
\end{gather*}

\textbf{3. Quasi-triangle Inequality}

\begin{gather*}
    \lVert f + g \rVert_{p, \infty} = \sup_{\lambda > 0} \lambda \mu\{x: |f(x) + g(x)| > \lambda\}^{\frac{1}{p}}\\
    \leq \sup_{\lambda > 0} \lambda \left( \mu\{x: |f(x)| > \frac{\lambda}{2}\} + \mu\{x: |g(x)| > \frac{\lambda}{2}\} \right)^{\frac{1}{p}} \\
    = \sup_{\alpha > 0} 2\alpha \left( \mu\{x: |f(x)| > \alpha\} + \mu\{x: |g(x)| > \alpha\} \right)^{\frac{1}{p}} \quad \text{ substituting } \alpha = \frac{\lambda}{2}\\
    \leq 2 \sup_{\alpha > 0} \alpha \left( \left( \mu\{x: |f(x)| > \alpha\} \right)^{\frac{1}{p}} +  \left( \mu\{x: |g(x)| > \alpha\} \right)^{\frac{1}{p}} \right) \quad \text{Minkowski}\\
    \leq 2 \left( \lVert f \rVert_{p, \infty} + \lVert g \rVert_{p, \infty} \right)
\end{gather*}
Therefore, weak p norm satisfies quasi triangle inequality with $C = 2$.
\\~

\textbf{Problem 4. } 

$\{T_t\}$ be a family of linear operators on $L^p(\R)$ and $T^* f(x) = \sup_t |T_tf(x)|$ be their associated maximal operator. Now if we suppose that $T^*$ is weak $(p,q)$ where $1 \leq p,q \leq \infty$, then we want to show that the set
\begin{gather*}
    \{f \in L^p(\R): \lim_{t \to t_0} T_tf(x) \text{ exists a.e.}\}
\end{gather*}
is closed in $L^p(\R)$.
\\~

Following the hint, we note that the limit exists iff $\limsup_{t\to t_0} T_tf - \liminf_{t \to t_0}T_t f = 0$. Now consider, a sequence of functions $f_n \in \{f \in L^p(\R): \lim_{t \to t_0} T_tf(x) \text{ exists a.e.}\}$ s.t $f_n \to f$ in $L^p$. Define:
\begin{gather*}
    S_f(x) = \limsup_{t\to t_0} T_tf - \liminf_{t \to t_0}T_t f
\end{gather*}
Then, clearly $S_{f_n}(x) = 0, \forall n$.
\\~

Now we follow a similar proof like Theorem 2 in Lecture 10. First we need to upper bound the measures of our sets:
\begin{gather*}
    \lambda>0,\, \mu\{x\in \R: S_f(x) > \lambda\}
\end{gather*}
with something more familiar that can be bounded.

\begin{gather*}
    |T_af(x) - T_af_n(x)| \leq T^* (f-f_n)(x)\\
    |T_bf(x) - T_bf_n(x)| \leq T^* (f-f_n)(x)
\end{gather*}
Now
\begin{gather*}
    \limsup_{t\to t_0} T_{a}f(x) - \liminf_{t \to t_0} T_{b}f(x) \\
    = \limsup_{t\to t_0} T_{a}(f(x) - f_n(x)) - \liminf_{t \to t_0} T_{b}(f(x) - f_n(x)) +\\ - \limsup_{t\to t_0} T_{a}f_n(x) + \liminf_{t \to t_0} T_{b}f_n(x)\\
    = \limsup_{t\to t_0} T_{a}(f(x) - f_n(x)) - \liminf_{t \to t_0} T_{b}(f(x) - f_n(x)) \\
    \leq |T_{a}(f(x) - f_n(x))| + |T_{b}(f(x) - f_n(x))| \\
    \leq 2 T^* (f-f_n)(x)
\end{gather*}
Now, we have our more familiar set!
\begin{gather*}
    \lambda > 0 : \mu\{x\in \R: S_f(x) > \lambda\} \subset \mu\{x\in \R: 2 T^* (f-f_n)(x) > \lambda\}
\end{gather*}
By definition of weak $(p,q)$, we have:
\begin{gather*}
    \mu\{x\in \R: T^* (f-f_n)(x) > \lambda/2\} \lesssim \frac{1}{\lambda^q} \lVert f - f_n\rVert_{L^p}^q \to 0\\
    \implies \mu\{x\in \R: S_f(x) > \lambda\} = 0
\end{gather*}
Therefore, we can conclude that $\lim_{t\to t_0} T_t f$ exists a.e. Hence it is in the set, implying that the set is closed.
\\~



\textbf{Problem 5. } 
\begin{gather*}
    Mf(x) = \sup_{r > 0} \frac{1}{\mu(B(x,r))} \int_{B(x,r)} |f(y)|dy\\
    M'f(x) = \sup_{I \ni x} \frac{1}{\mu(I)} \int_{I} |f(y)|dy \quad \forall f \in L_{loc}^1(\R), \forall x \in \R
\end{gather*}

\textbf{Claim: } $cMf(x) \leq M'f(x)$
\\~

This direction of the proof is trivial. Every ball $B(x,r)$ is an interval centered at $x$ and any such interval is included in the search for the supremum on $M'f(x)$. Therefore, $Mf(x)$ is a supremum taken over a subset of the intervals considered in $M'f(x)$. Hence $Mf(x) \leq M'f(x)$. This proves the lower bound with constant $c=1$.
\\~

\textbf{Claim: } $M'f(x) \leq C Mf(x)$
\\~

$\forall I \ni x$ let $r = \mu(I)$, then $I \subset B(x,r)$
\begin{gather*}
    \frac{1}{\mu(I)} \int_{I} |f(y)|dy \leq \frac{1}{\mu(I)} \int_{B(x,r)} |f(y)|dy \\
    = \frac{1}{r} \int_{B(x,r)} |f(y)|dy \\
    = \frac{2}{\mu(B(x, r))} \int_{B(x,r)} |f(y)|dy  \quad \text{ since in } \R, \mu(B(x, r)) = 2r\\
    \leq 2 \sup_{r > 0} \frac{1}{\mu(B(x,r))} \int_{B(x,r)} |f(y)|dy = 2 Mf(x)
\end{gather*}
Therefore,
\begin{gather*}
    M'f(x) = \sup_{I\ni x} \frac{1}{\mu(I)} \int_{I} |f(y)|dy \leq 2 Mf(x)
\end{gather*}
Therefore, we found an upper bound with $C = 2$.
\\~

\textbf{Problem 6. }

\textbf{Fun fact: } Being from a CS background, I can't help but notice the similarities in thought process between covering lemmas and a certain class of greedy algorithms, like interval scheduling, etc. \href{https://www.cs.princeton.edu/~wayne/kleinberg-tardos/pearson/04GreedyAlgorithms-2x2.pdf}{interval scheduling link}.
\\~


Since, $K$ is compact, there exists a finite subcollection of open intervals in $\R$ such that,
\begin{gather*}
    K \subset \bigcup_{j=1}^{M} I_j
\end{gather*}
Now, we claim that we can select a subset of these intervals further such that we have a set that covers each point in $\R$ almost 2 times.

Taking inspiration from interval scheduling problems in algorithms, we propose a method to construct this subset of intervals here:
\begin{enumerate}
    \item Since, K is closed and bounded, let $x_1 = \min K$. Choose $I_1 \ni x_1$ as the interval with the highest right endpoint and interval size as the second priority.
    
    \item Let $c$ be the number of intervals chosen. If $c \geq 1$ and $I_1, I_2, \dots, I_c$ have been chosen, but $K \backslash \cup_{i=1}^c I_i \neq \phi$, then define $x_{c+1} = \min \{K \backslash \cup_{i=1}^c I_i\}$.
    
    \item Same as before, choose $I_{c+1} \ni x_{c+1}$ such that it has the highest right endpoint and interval size as the second priority. Continue from step 2.

    \item Since the number of intervals is finite (=M), this loop must terminate. This will yield a finite subcover such that $K \subset \cup_{j=1}^N I_j$.
\end{enumerate}
Now if we look at step 3 carefully, we should notice that interval $I_{c+1}$ cannot cover $x_{c}$, otherwise, $I_{c+1}$ would get chosen in the previous iteration. Hence, every $I_{c+1} \bigcap \cup_{j=1}^{c-1}I_j = \phi$. Hence the sets of intervals $\{I_{2k}\}_{k=1}^{N//2}$ and $\{I_{2k+1}\}_{k=0}^{(N-1)//2}$ are pairwise disjoint. Hence,

\begin{gather*}
    \sum_{j=1}^{N}\chi I_j(x) \leq 2, \, \forall x\in \R\\
    \text{And } K \subset \bigcup_{j=1}^N I_j \text{ by construction}
\end{gather*}




\end{document} 